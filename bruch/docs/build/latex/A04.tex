% Generated by Sphinx.
\def\sphinxdocclass{report}
\newif\ifsphinxKeepOldNames \sphinxKeepOldNamestrue
\documentclass[letterpaper,10pt,english]{sphinxmanual}
\usepackage{iftex}

\ifPDFTeX
  \usepackage[utf8]{inputenc}
\fi
\ifdefined\DeclareUnicodeCharacter
  \DeclareUnicodeCharacter{00A0}{\nobreakspace}
\fi
\usepackage{cmap}
\usepackage[T1]{fontenc}
\usepackage{amsmath,amssymb,amstext}
\usepackage{babel}
\usepackage{times}
\usepackage[Bjarne]{fncychap}
\usepackage{longtable}
\usepackage{sphinx}
\usepackage{multirow}
\usepackage{eqparbox}


\addto\captionsenglish{\renewcommand{\figurename}{Fig.\@ }}
\addto\captionsenglish{\renewcommand{\tablename}{Table }}
\SetupFloatingEnvironment{literal-block}{name=Listing }

\addto\extrasenglish{\def\pageautorefname{page}}

\setcounter{tocdepth}{1}


\title{Bruch Documentation}
\date{Oct 23, 2016}
\release{1.0}
\author{Marvin Ertl}
\newcommand{\sphinxlogo}{}
\renewcommand{\releasename}{Release}
\makeindex

\makeatletter
\def\PYG@reset{\let\PYG@it=\relax \let\PYG@bf=\relax%
    \let\PYG@ul=\relax \let\PYG@tc=\relax%
    \let\PYG@bc=\relax \let\PYG@ff=\relax}
\def\PYG@tok#1{\csname PYG@tok@#1\endcsname}
\def\PYG@toks#1+{\ifx\relax#1\empty\else%
    \PYG@tok{#1}\expandafter\PYG@toks\fi}
\def\PYG@do#1{\PYG@bc{\PYG@tc{\PYG@ul{%
    \PYG@it{\PYG@bf{\PYG@ff{#1}}}}}}}
\def\PYG#1#2{\PYG@reset\PYG@toks#1+\relax+\PYG@do{#2}}

\expandafter\def\csname PYG@tok@gd\endcsname{\def\PYG@tc##1{\textcolor[rgb]{0.63,0.00,0.00}{##1}}}
\expandafter\def\csname PYG@tok@gu\endcsname{\let\PYG@bf=\textbf\def\PYG@tc##1{\textcolor[rgb]{0.50,0.00,0.50}{##1}}}
\expandafter\def\csname PYG@tok@gt\endcsname{\def\PYG@tc##1{\textcolor[rgb]{0.00,0.27,0.87}{##1}}}
\expandafter\def\csname PYG@tok@gs\endcsname{\let\PYG@bf=\textbf}
\expandafter\def\csname PYG@tok@gr\endcsname{\def\PYG@tc##1{\textcolor[rgb]{1.00,0.00,0.00}{##1}}}
\expandafter\def\csname PYG@tok@cm\endcsname{\let\PYG@it=\textit\def\PYG@tc##1{\textcolor[rgb]{0.25,0.50,0.56}{##1}}}
\expandafter\def\csname PYG@tok@vg\endcsname{\def\PYG@tc##1{\textcolor[rgb]{0.73,0.38,0.84}{##1}}}
\expandafter\def\csname PYG@tok@vi\endcsname{\def\PYG@tc##1{\textcolor[rgb]{0.73,0.38,0.84}{##1}}}
\expandafter\def\csname PYG@tok@mh\endcsname{\def\PYG@tc##1{\textcolor[rgb]{0.13,0.50,0.31}{##1}}}
\expandafter\def\csname PYG@tok@cs\endcsname{\def\PYG@tc##1{\textcolor[rgb]{0.25,0.50,0.56}{##1}}\def\PYG@bc##1{\setlength{\fboxsep}{0pt}\colorbox[rgb]{1.00,0.94,0.94}{\strut ##1}}}
\expandafter\def\csname PYG@tok@ge\endcsname{\let\PYG@it=\textit}
\expandafter\def\csname PYG@tok@vc\endcsname{\def\PYG@tc##1{\textcolor[rgb]{0.73,0.38,0.84}{##1}}}
\expandafter\def\csname PYG@tok@il\endcsname{\def\PYG@tc##1{\textcolor[rgb]{0.13,0.50,0.31}{##1}}}
\expandafter\def\csname PYG@tok@go\endcsname{\def\PYG@tc##1{\textcolor[rgb]{0.20,0.20,0.20}{##1}}}
\expandafter\def\csname PYG@tok@cp\endcsname{\def\PYG@tc##1{\textcolor[rgb]{0.00,0.44,0.13}{##1}}}
\expandafter\def\csname PYG@tok@gi\endcsname{\def\PYG@tc##1{\textcolor[rgb]{0.00,0.63,0.00}{##1}}}
\expandafter\def\csname PYG@tok@gh\endcsname{\let\PYG@bf=\textbf\def\PYG@tc##1{\textcolor[rgb]{0.00,0.00,0.50}{##1}}}
\expandafter\def\csname PYG@tok@ni\endcsname{\let\PYG@bf=\textbf\def\PYG@tc##1{\textcolor[rgb]{0.84,0.33,0.22}{##1}}}
\expandafter\def\csname PYG@tok@nl\endcsname{\let\PYG@bf=\textbf\def\PYG@tc##1{\textcolor[rgb]{0.00,0.13,0.44}{##1}}}
\expandafter\def\csname PYG@tok@nn\endcsname{\let\PYG@bf=\textbf\def\PYG@tc##1{\textcolor[rgb]{0.05,0.52,0.71}{##1}}}
\expandafter\def\csname PYG@tok@no\endcsname{\def\PYG@tc##1{\textcolor[rgb]{0.38,0.68,0.84}{##1}}}
\expandafter\def\csname PYG@tok@na\endcsname{\def\PYG@tc##1{\textcolor[rgb]{0.25,0.44,0.63}{##1}}}
\expandafter\def\csname PYG@tok@nb\endcsname{\def\PYG@tc##1{\textcolor[rgb]{0.00,0.44,0.13}{##1}}}
\expandafter\def\csname PYG@tok@nc\endcsname{\let\PYG@bf=\textbf\def\PYG@tc##1{\textcolor[rgb]{0.05,0.52,0.71}{##1}}}
\expandafter\def\csname PYG@tok@nd\endcsname{\let\PYG@bf=\textbf\def\PYG@tc##1{\textcolor[rgb]{0.33,0.33,0.33}{##1}}}
\expandafter\def\csname PYG@tok@ne\endcsname{\def\PYG@tc##1{\textcolor[rgb]{0.00,0.44,0.13}{##1}}}
\expandafter\def\csname PYG@tok@nf\endcsname{\def\PYG@tc##1{\textcolor[rgb]{0.02,0.16,0.49}{##1}}}
\expandafter\def\csname PYG@tok@si\endcsname{\let\PYG@it=\textit\def\PYG@tc##1{\textcolor[rgb]{0.44,0.63,0.82}{##1}}}
\expandafter\def\csname PYG@tok@s2\endcsname{\def\PYG@tc##1{\textcolor[rgb]{0.25,0.44,0.63}{##1}}}
\expandafter\def\csname PYG@tok@nt\endcsname{\let\PYG@bf=\textbf\def\PYG@tc##1{\textcolor[rgb]{0.02,0.16,0.45}{##1}}}
\expandafter\def\csname PYG@tok@nv\endcsname{\def\PYG@tc##1{\textcolor[rgb]{0.73,0.38,0.84}{##1}}}
\expandafter\def\csname PYG@tok@s1\endcsname{\def\PYG@tc##1{\textcolor[rgb]{0.25,0.44,0.63}{##1}}}
\expandafter\def\csname PYG@tok@ch\endcsname{\let\PYG@it=\textit\def\PYG@tc##1{\textcolor[rgb]{0.25,0.50,0.56}{##1}}}
\expandafter\def\csname PYG@tok@m\endcsname{\def\PYG@tc##1{\textcolor[rgb]{0.13,0.50,0.31}{##1}}}
\expandafter\def\csname PYG@tok@gp\endcsname{\let\PYG@bf=\textbf\def\PYG@tc##1{\textcolor[rgb]{0.78,0.36,0.04}{##1}}}
\expandafter\def\csname PYG@tok@sh\endcsname{\def\PYG@tc##1{\textcolor[rgb]{0.25,0.44,0.63}{##1}}}
\expandafter\def\csname PYG@tok@ow\endcsname{\let\PYG@bf=\textbf\def\PYG@tc##1{\textcolor[rgb]{0.00,0.44,0.13}{##1}}}
\expandafter\def\csname PYG@tok@sx\endcsname{\def\PYG@tc##1{\textcolor[rgb]{0.78,0.36,0.04}{##1}}}
\expandafter\def\csname PYG@tok@bp\endcsname{\def\PYG@tc##1{\textcolor[rgb]{0.00,0.44,0.13}{##1}}}
\expandafter\def\csname PYG@tok@c1\endcsname{\let\PYG@it=\textit\def\PYG@tc##1{\textcolor[rgb]{0.25,0.50,0.56}{##1}}}
\expandafter\def\csname PYG@tok@o\endcsname{\def\PYG@tc##1{\textcolor[rgb]{0.40,0.40,0.40}{##1}}}
\expandafter\def\csname PYG@tok@kc\endcsname{\let\PYG@bf=\textbf\def\PYG@tc##1{\textcolor[rgb]{0.00,0.44,0.13}{##1}}}
\expandafter\def\csname PYG@tok@c\endcsname{\let\PYG@it=\textit\def\PYG@tc##1{\textcolor[rgb]{0.25,0.50,0.56}{##1}}}
\expandafter\def\csname PYG@tok@mf\endcsname{\def\PYG@tc##1{\textcolor[rgb]{0.13,0.50,0.31}{##1}}}
\expandafter\def\csname PYG@tok@err\endcsname{\def\PYG@bc##1{\setlength{\fboxsep}{0pt}\fcolorbox[rgb]{1.00,0.00,0.00}{1,1,1}{\strut ##1}}}
\expandafter\def\csname PYG@tok@mb\endcsname{\def\PYG@tc##1{\textcolor[rgb]{0.13,0.50,0.31}{##1}}}
\expandafter\def\csname PYG@tok@ss\endcsname{\def\PYG@tc##1{\textcolor[rgb]{0.32,0.47,0.09}{##1}}}
\expandafter\def\csname PYG@tok@sr\endcsname{\def\PYG@tc##1{\textcolor[rgb]{0.14,0.33,0.53}{##1}}}
\expandafter\def\csname PYG@tok@mo\endcsname{\def\PYG@tc##1{\textcolor[rgb]{0.13,0.50,0.31}{##1}}}
\expandafter\def\csname PYG@tok@kd\endcsname{\let\PYG@bf=\textbf\def\PYG@tc##1{\textcolor[rgb]{0.00,0.44,0.13}{##1}}}
\expandafter\def\csname PYG@tok@mi\endcsname{\def\PYG@tc##1{\textcolor[rgb]{0.13,0.50,0.31}{##1}}}
\expandafter\def\csname PYG@tok@kn\endcsname{\let\PYG@bf=\textbf\def\PYG@tc##1{\textcolor[rgb]{0.00,0.44,0.13}{##1}}}
\expandafter\def\csname PYG@tok@cpf\endcsname{\let\PYG@it=\textit\def\PYG@tc##1{\textcolor[rgb]{0.25,0.50,0.56}{##1}}}
\expandafter\def\csname PYG@tok@kr\endcsname{\let\PYG@bf=\textbf\def\PYG@tc##1{\textcolor[rgb]{0.00,0.44,0.13}{##1}}}
\expandafter\def\csname PYG@tok@s\endcsname{\def\PYG@tc##1{\textcolor[rgb]{0.25,0.44,0.63}{##1}}}
\expandafter\def\csname PYG@tok@kp\endcsname{\def\PYG@tc##1{\textcolor[rgb]{0.00,0.44,0.13}{##1}}}
\expandafter\def\csname PYG@tok@w\endcsname{\def\PYG@tc##1{\textcolor[rgb]{0.73,0.73,0.73}{##1}}}
\expandafter\def\csname PYG@tok@kt\endcsname{\def\PYG@tc##1{\textcolor[rgb]{0.56,0.13,0.00}{##1}}}
\expandafter\def\csname PYG@tok@sc\endcsname{\def\PYG@tc##1{\textcolor[rgb]{0.25,0.44,0.63}{##1}}}
\expandafter\def\csname PYG@tok@sb\endcsname{\def\PYG@tc##1{\textcolor[rgb]{0.25,0.44,0.63}{##1}}}
\expandafter\def\csname PYG@tok@k\endcsname{\let\PYG@bf=\textbf\def\PYG@tc##1{\textcolor[rgb]{0.00,0.44,0.13}{##1}}}
\expandafter\def\csname PYG@tok@se\endcsname{\let\PYG@bf=\textbf\def\PYG@tc##1{\textcolor[rgb]{0.25,0.44,0.63}{##1}}}
\expandafter\def\csname PYG@tok@sd\endcsname{\let\PYG@it=\textit\def\PYG@tc##1{\textcolor[rgb]{0.25,0.44,0.63}{##1}}}

\def\PYGZbs{\char`\\}
\def\PYGZus{\char`\_}
\def\PYGZob{\char`\{}
\def\PYGZcb{\char`\}}
\def\PYGZca{\char`\^}
\def\PYGZam{\char`\&}
\def\PYGZlt{\char`\<}
\def\PYGZgt{\char`\>}
\def\PYGZsh{\char`\#}
\def\PYGZpc{\char`\%}
\def\PYGZdl{\char`\$}
\def\PYGZhy{\char`\-}
\def\PYGZsq{\char`\'}
\def\PYGZdq{\char`\"}
\def\PYGZti{\char`\~}
% for compatibility with earlier versions
\def\PYGZat{@}
\def\PYGZlb{[}
\def\PYGZrb{]}
\makeatother

\renewcommand\PYGZsq{\textquotesingle}

\begin{document}

\maketitle
\tableofcontents
\phantomsection\label{index::doc}


Contents:


\chapter{Bruch}
\label{bruch:welcome-to-bruch-s-documentation}\label{bruch:module-bruch}\label{bruch::doc}\label{bruch:bruch}\index{bruch (module)}\index{Bruch (class in bruch)}

\begin{fulllineitems}
\phantomsection\label{bruch:bruch.Bruch}\pysiglinewithargsret{\sphinxstrong{class }\sphinxcode{bruch.}\sphinxbfcode{Bruch}}{\emph{*args}}{}
Bases: \sphinxcode{object}

The class Bruch represents a fraction.
Nearly all operator of this class are overloaded.
\index{\_Bruch\_\_makeBruch() (bruch.Bruch static method)}

\begin{fulllineitems}
\phantomsection\label{bruch:bruch.Bruch._Bruch__makeBruch}\pysiglinewithargsret{\sphinxstrong{static }\sphinxbfcode{\_Bruch\_\_makeBruch}}{\emph{value}}{}
Creates a fraction
:param value: int or a fraction
:return: a fraction

\end{fulllineitems}

\index{\_\_abs\_\_() (bruch.Bruch method)}

\begin{fulllineitems}
\phantomsection\label{bruch:bruch.Bruch.__abs__}\pysiglinewithargsret{\sphinxbfcode{\_\_abs\_\_}}{}{}
Returns absolute value of Bruch
:return: float

\end{fulllineitems}

\index{\_\_add\_\_() (bruch.Bruch method)}

\begin{fulllineitems}
\phantomsection\label{bruch:bruch.Bruch.__add__}\pysiglinewithargsret{\sphinxbfcode{\_\_add\_\_}}{\emph{other}}{}
Adds other to self
:param other: int or a fraction
:return: a fraction

\end{fulllineitems}

\index{\_\_dict\_\_ (bruch.Bruch attribute)}

\begin{fulllineitems}
\phantomsection\label{bruch:bruch.Bruch.__dict__}\pysigline{\sphinxbfcode{\_\_dict\_\_}\sphinxstrong{ = dict\_proxy(\{`\_\_int\_\_': \textless{}function \_\_int\_\_ at 0x04395B70\textgreater{}, `\_\_module\_\_': `bruch.Bruch', `\_\_rtruediv\_\_': \textless{}function \_\_rtruediv\_\_ at 0x043952F0\textgreater{}, `\_\_isub\_\_': \textless{}function \_\_isub\_\_ at 0x04395AB0\textgreater{}, `\_\_str\_\_': \textless{}function \_\_str\_\_ at 0x04395DB0\textgreater{}, `\_\_radd\_\_': \textless{}function \_\_radd\_\_\textgreater{}, `\_\_dict\_\_': \textless{}attribute `\_\_dict\_\_' of `Bruch' objects\textgreater{}, `\_\_truediv\_\_': \textless{}function \_\_truediv\_\_ at 0x043953B0\textgreater{}, `\_\_rsub\_\_': \textless{}function \_\_rsub\_\_ at 0x04395A70\textgreater{}, `\_\_rdiv\_\_': \textless{}function \_\_rdiv\_\_ at 0x04395AF0\textgreater{}, `\_\_rmul\_\_': \textless{}function \_\_rmul\_\_ at 0x04395C70\textgreater{}, `\_Bruch\_\_makeBruch': \textless{}staticmethod object at 0x0428AB70\textgreater{}, `\_\_weakref\_\_': \textless{}attribute `\_\_weakref\_\_' of `Bruch' objects\textgreater{}, `\_\_lt\_\_': \textless{}function \_\_lt\_\_ at 0x04395EF0\textgreater{}, `\_\_init\_\_': \textless{}function \_\_init\_\_ at 0x04395BF0\textgreater{}, `\_\_float\_\_': \textless{}function \_\_float\_\_ at 0x04395B30\textgreater{}, `\_\_abs\_\_': \textless{}function \_\_abs\_\_ at 0x04395C30\textgreater{}, `\_\_doc\_\_': `\textbackslash{}n    The class Bruch represents a fraction.\textbackslash{}n    Nearly all operator of this class are overloaded.\textbackslash{}n    `, `\_\_itruediv\_\_': \textless{}function \_\_itruediv\_\_\textgreater{}, `\_\_mul\_\_': \textless{}function \_\_mul\_\_ at 0x04395F30\textgreater{}, `\_\_ne\_\_': \textless{}function \_\_ne\_\_ at 0x043957F0\textgreater{}, `\_\_invert\_\_': \textless{}function \_\_invert\_\_ at 0x04395D30\textgreater{}, `\_\_iter\_\_': \textless{}function \_\_iter\_\_\textgreater{}, `\_\_pow\_\_': \textless{}function \_\_pow\_\_ at 0x04395CF0\textgreater{}, `\_\_add\_\_': \textless{}function \_\_add\_\_ at 0x04395F70\textgreater{}, `\_\_gt\_\_': \textless{}function \_\_gt\_\_ at 0x04395E70\textgreater{}, `\_\_eq\_\_': \textless{}function \_\_eq\_\_ at 0x04395DF0\textgreater{}, `\_\_imul\_\_': \textless{}function \_\_imul\_\_\textgreater{}, `\_\_iadd\_\_': \textless{}function \_\_iadd\_\_ at 0x04395BB0\textgreater{}, `\_\_div\_\_': \textless{}function \_\_div\_\_ at 0x043958B0\textgreater{}, `\_\_le\_\_': \textless{}function \_\_le\_\_ at 0x04395EB0\textgreater{}, `\_\_neg\_\_': \textless{}function \_\_neg\_\_ at 0x04395D70\textgreater{}, `\_\_sub\_\_': \textless{}function \_\_sub\_\_ at 0x04395A30\textgreater{}, `\_\_ge\_\_': \textless{}function \_\_ge\_\_ at 0x04395E30\textgreater{}\})}}
\end{fulllineitems}

\index{\_\_div\_\_() (bruch.Bruch method)}

\begin{fulllineitems}
\phantomsection\label{bruch:bruch.Bruch.__div__}\pysiglinewithargsret{\sphinxbfcode{\_\_div\_\_}}{\emph{other}}{}
Divide ergebnis through other
:param other: int or a fraction
:return: float

\end{fulllineitems}

\index{\_\_eq\_\_() (bruch.Bruch method)}

\begin{fulllineitems}
\phantomsection\label{bruch:bruch.Bruch.__eq__}\pysiglinewithargsret{\sphinxbfcode{\_\_eq\_\_}}{\emph{other}}{}
Test if self is equal to other
:param other: int or a fraction
:return: boolean

\end{fulllineitems}

\index{\_\_float\_\_() (bruch.Bruch method)}

\begin{fulllineitems}
\phantomsection\label{bruch:bruch.Bruch.__float__}\pysiglinewithargsret{\sphinxbfcode{\_\_float\_\_}}{}{}
Returns float of Bruch
:return: float

\end{fulllineitems}

\index{\_\_ge\_\_() (bruch.Bruch method)}

\begin{fulllineitems}
\phantomsection\label{bruch:bruch.Bruch.__ge__}\pysiglinewithargsret{\sphinxbfcode{\_\_ge\_\_}}{\emph{other}}{}
Test if self is equal/bigger to other
:param other: int or a fraction
:return: boolean

\end{fulllineitems}

\index{\_\_gt\_\_() (bruch.Bruch method)}

\begin{fulllineitems}
\phantomsection\label{bruch:bruch.Bruch.__gt__}\pysiglinewithargsret{\sphinxbfcode{\_\_gt\_\_}}{\emph{other}}{}
Test if self is bigger than other
:param other: int or a fraction
:return: boolean

\end{fulllineitems}

\index{\_\_iadd\_\_() (bruch.Bruch method)}

\begin{fulllineitems}
\phantomsection\label{bruch:bruch.Bruch.__iadd__}\pysiglinewithargsret{\sphinxbfcode{\_\_iadd\_\_}}{\emph{other}}{}
Adds self to other
:param other: int or a fraction
:return: float

\end{fulllineitems}

\index{\_\_imul\_\_() (bruch.Bruch method)}

\begin{fulllineitems}
\phantomsection\label{bruch:bruch.Bruch.__imul__}\pysiglinewithargsret{\sphinxbfcode{\_\_imul\_\_}}{\emph{other}}{}
Multiply self with other
:param other: int or a fraction
:return: float

\end{fulllineitems}

\index{\_\_init\_\_() (bruch.Bruch method)}

\begin{fulllineitems}
\phantomsection\label{bruch:bruch.Bruch.__init__}\pysiglinewithargsret{\sphinxbfcode{\_\_init\_\_}}{\emph{*args}}{}
Create a fraction or throw a exception if the parameters are not correct, for example Zero Division, Type Error.
:param args: parameters for the fraction can be int, float or Bruch

\end{fulllineitems}

\index{\_\_int\_\_() (bruch.Bruch method)}

\begin{fulllineitems}
\phantomsection\label{bruch:bruch.Bruch.__int__}\pysiglinewithargsret{\sphinxbfcode{\_\_int\_\_}}{}{}
Returns int of Bruch
:return: int

\end{fulllineitems}

\index{\_\_invert\_\_() (bruch.Bruch method)}

\begin{fulllineitems}
\phantomsection\label{bruch:bruch.Bruch.__invert__}\pysiglinewithargsret{\sphinxbfcode{\_\_invert\_\_}}{}{}
Divide nenner through zaehler
:return: float

\end{fulllineitems}

\index{\_\_isub\_\_() (bruch.Bruch method)}

\begin{fulllineitems}
\phantomsection\label{bruch:bruch.Bruch.__isub__}\pysiglinewithargsret{\sphinxbfcode{\_\_isub\_\_}}{\emph{other}}{}
Substract other from self
:param other: int or a fraction
:return: float

\end{fulllineitems}

\index{\_\_iter\_\_() (bruch.Bruch method)}

\begin{fulllineitems}
\phantomsection\label{bruch:bruch.Bruch.__iter__}\pysiglinewithargsret{\sphinxbfcode{\_\_iter\_\_}}{}{}
Iterator of the fraction
:return: iterator

\end{fulllineitems}

\index{\_\_itruediv\_\_() (bruch.Bruch method)}

\begin{fulllineitems}
\phantomsection\label{bruch:bruch.Bruch.__itruediv__}\pysiglinewithargsret{\sphinxbfcode{\_\_itruediv\_\_}}{\emph{other}}{}
Divide self through other
:param other: int or a fraction
:return: float

\end{fulllineitems}

\index{\_\_le\_\_() (bruch.Bruch method)}

\begin{fulllineitems}
\phantomsection\label{bruch:bruch.Bruch.__le__}\pysiglinewithargsret{\sphinxbfcode{\_\_le\_\_}}{\emph{other}}{}
Test if self is smaller/equal to other
:param other: int or a fraction
:return: boolean

\end{fulllineitems}

\index{\_\_lt\_\_() (bruch.Bruch method)}

\begin{fulllineitems}
\phantomsection\label{bruch:bruch.Bruch.__lt__}\pysiglinewithargsret{\sphinxbfcode{\_\_lt\_\_}}{\emph{other}}{}
Test if self is smaller than other
:param other: int or a fraction
:return: boolean

\end{fulllineitems}

\index{\_\_module\_\_ (bruch.Bruch attribute)}

\begin{fulllineitems}
\phantomsection\label{bruch:bruch.Bruch.__module__}\pysigline{\sphinxbfcode{\_\_module\_\_}\sphinxstrong{ = `bruch.Bruch'}}
\end{fulllineitems}

\index{\_\_mul\_\_() (bruch.Bruch method)}

\begin{fulllineitems}
\phantomsection\label{bruch:bruch.Bruch.__mul__}\pysiglinewithargsret{\sphinxbfcode{\_\_mul\_\_}}{\emph{other}}{}
Multiply self with other
:param other: int or a fraction
:return: float

\end{fulllineitems}

\index{\_\_ne\_\_() (bruch.Bruch method)}

\begin{fulllineitems}
\phantomsection\label{bruch:bruch.Bruch.__ne__}\pysiglinewithargsret{\sphinxbfcode{\_\_ne\_\_}}{\emph{other}}{}
Test if self is not equal to other
:param other: int or a fraction
:return: boolean

\end{fulllineitems}

\index{\_\_neg\_\_() (bruch.Bruch method)}

\begin{fulllineitems}
\phantomsection\label{bruch:bruch.Bruch.__neg__}\pysiglinewithargsret{\sphinxbfcode{\_\_neg\_\_}}{}{}
Divide zaehler through zaehler and take it -1 times
:return: float

\end{fulllineitems}

\index{\_\_pow\_\_() (bruch.Bruch method)}

\begin{fulllineitems}
\phantomsection\label{bruch:bruch.Bruch.__pow__}\pysiglinewithargsret{\sphinxbfcode{\_\_pow\_\_}}{\emph{power}, \emph{modulo=None}}{}
Takes self up to tje power
:param power: int
:param modulo: None
:return: a fraction

\end{fulllineitems}

\index{\_\_radd\_\_() (bruch.Bruch method)}

\begin{fulllineitems}
\phantomsection\label{bruch:bruch.Bruch.__radd__}\pysiglinewithargsret{\sphinxbfcode{\_\_radd\_\_}}{\emph{other}}{}
Adds self to other
:param other: int or a fraction
:return: float

\end{fulllineitems}

\index{\_\_rdiv\_\_() (bruch.Bruch method)}

\begin{fulllineitems}
\phantomsection\label{bruch:bruch.Bruch.__rdiv__}\pysiglinewithargsret{\sphinxbfcode{\_\_rdiv\_\_}}{\emph{other}}{}
Divide self.zaehler through other
:param other: int or a fration
:return: float

\end{fulllineitems}

\index{\_\_rmul\_\_() (bruch.Bruch method)}

\begin{fulllineitems}
\phantomsection\label{bruch:bruch.Bruch.__rmul__}\pysiglinewithargsret{\sphinxbfcode{\_\_rmul\_\_}}{\emph{other}}{}
Multiply other with self
:param other: int or a fraction
:return: float

\end{fulllineitems}

\index{\_\_rsub\_\_() (bruch.Bruch method)}

\begin{fulllineitems}
\phantomsection\label{bruch:bruch.Bruch.__rsub__}\pysiglinewithargsret{\sphinxbfcode{\_\_rsub\_\_}}{\emph{other}}{}
Substract self from other
:param other: int or a fraction
:return: float

\end{fulllineitems}

\index{\_\_rtruediv\_\_() (bruch.Bruch method)}

\begin{fulllineitems}
\phantomsection\label{bruch:bruch.Bruch.__rtruediv__}\pysiglinewithargsret{\sphinxbfcode{\_\_rtruediv\_\_}}{\emph{other}}{}
Multiply self to not other
:param other: int or a fraction
:return: float

\end{fulllineitems}

\index{\_\_str\_\_() (bruch.Bruch method)}

\begin{fulllineitems}
\phantomsection\label{bruch:bruch.Bruch.__str__}\pysiglinewithargsret{\sphinxbfcode{\_\_str\_\_}}{}{}
Represented the fraction as (zaehler/nenner)
:return: str

\end{fulllineitems}

\index{\_\_sub\_\_() (bruch.Bruch method)}

\begin{fulllineitems}
\phantomsection\label{bruch:bruch.Bruch.__sub__}\pysiglinewithargsret{\sphinxbfcode{\_\_sub\_\_}}{\emph{other}}{}
Substract other from self
:param other: int or a fraction
:return: float

\end{fulllineitems}

\index{\_\_truediv\_\_() (bruch.Bruch method)}

\begin{fulllineitems}
\phantomsection\label{bruch:bruch.Bruch.__truediv__}\pysiglinewithargsret{\sphinxbfcode{\_\_truediv\_\_}}{\emph{other}}{}
Multiply self to not other
:param other: int or a fraction
:return: float

\end{fulllineitems}

\index{\_\_weakref\_\_ (bruch.Bruch attribute)}

\begin{fulllineitems}
\phantomsection\label{bruch:bruch.Bruch.__weakref__}\pysigline{\sphinxbfcode{\_\_weakref\_\_}}
list of weak references to the object (if defined)

\end{fulllineitems}


\end{fulllineitems}



\chapter{Indices and tables}
\label{index:indices-and-tables}\begin{itemize}
\item {} 
\DUrole{xref,std,std-ref}{genindex}

\item {} 
\DUrole{xref,std,std-ref}{modindex}

\item {} 
\DUrole{xref,std,std-ref}{search}

\end{itemize}


\renewcommand{\indexname}{Python Module Index}
\begin{theindex}
\def\bigletter#1{{\Large\sffamily#1}\nopagebreak\vspace{1mm}}
\bigletter{b}
\item {\texttt{bruch}}, \pageref{bruch:module-bruch}
\end{theindex}

\renewcommand{\indexname}{Index}
\printindex
\end{document}
